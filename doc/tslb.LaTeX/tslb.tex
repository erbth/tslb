% !TeX spellcheck = en_US
\documentclass[a4paper]{scrartcl}
\usepackage[left=2cm, right=2cm, top=2cm, bottom=2cm]{geometry}
\usepackage{lmodern}
\usepackage[T1]{fontenc}
\usepackage[utf8]{inputenc}
\usepackage[english]{babel}
\usepackage{float}
\usepackage{tabularx}
\usepackage{ltablex}
\usepackage[dvipsnames]{xcolor}
\usepackage{dirtree}
\usepackage[hidelinks]{hyperref}
\usepackage{algpseudocode}
\usepackage{algorithm}
\usepackage{algorithmicx}
\usepackage{tikz}
\usetikzlibrary{automata, positioning, arrows, shapes.geometric}
\usepackage{listings}
\usepackage{lstautogobble}

\newcommand{\file}[1]{\texttt{#1}}
\newcommand{\program}[1]{\textbf{#1}}
\newcommand{\variable}[1]{'\texttt{#1}'}
\newcommand{\module}[1]{\texttt{#1}}
\newcommand{\script}[1]{\texttt{#1}}

\newcommand{\python}[1]{\texttt{`#1'}}

\newcommand{\green}[1]{\textcolor{green}{#1}}

\lstset{
	basicstyle=\ttfamily,
	tabsize=4,
	numbers=left,
	columns=fullflexible,
	autogobble
}

\title{The TSClient LEGACY Build System}
\subtitle{Developer's documentation}
\author{Thomas Erbesdobler <t.erbesdobler@gmx.de>}

\begin{document}
	\maketitle
	\tableofcontents
	
	\pagebreak
	
	\section{Source package attributes}
	\label{sec:source_package_attributes}
	
	\begin{tabularx}{\textwidth}{llX}
		Command & Typical type & Description \\
		\hline
		upstream\_source\_url & str & URL from which the upstream source tarball can be downloaded \\\
	\end{tabularx}
	
	\subsection{Attributes of a specific version}
	\label{sec:source_package_version_attributes}
	
	\begin{tabularx}{\textwidth}{llX}
		Command & Typical type & Description \\
		\hline
		configure\_command & str & \\
		build\_command & str & \\
		install\_to\_destdir\_command & str & \\
		skip\_rdeps & str & Optionally skip adding rdeps (parsed into bool) \\
		disable\_python\_compileall & str & \\
		adapt\_command & str & \\
		additional\_file\_placement & List(Tuple(str, str|List(str))) & [(dst pkg name, regex | [regex])]\\

		additional\_rdeps & List(Tuple(str,DependencyList)) & \\
		remove\_rdeps & List(Tuple(str, str|List(str))) & bpkg -> fullmatch-regex; evaluated before \texttt{additional\_rdeps} \\
		file\_splitter & str|bytes & Custom executable to split files into packages, see section \ref{sec:file_splitter} \\
		
		cdeps & DependencyList & \\
		cdeps\_order\_only & List(str) & list of cdeps out of \texttt{cdeps} that shall not be installed \\
		tools & DependencyList &  Like cdeps, but only used for rootfs creation \\
		enabled & str & parsed into bool \\
		source\_archive & str|NoneType & \\
		unpack\_command & str & \\
		patch\_command & str & \\
		unpacked\_source\_directory & str & \\
		activated\_triggers & List(Tuple(str, str|List(str))) & \\
		activated\_triggers\_* & List(Tuple(str, str|List(str))) & \\
		interested\_triggers & List(Tuple(str, str|List(str))) & \\
		interested\_triggers\_* & List(Tuple(str, str|List(str))) & \\
		maintainer\_script\_* & str & \\
		disable\_dependency\_analyzer\_<...>\_for & List(str) & Disable the given dependency analyzer for the given binary packages (list of fullmatch-regexs) \\
		enable\_dependency\_analyzer\_<...>\_for & List(str) & Enable the given dependency analyzer for the given binary packages (if it is disabled by default) (list of fullmatch-regexs) \\
		strip\_skip\_paths & List(str) & List of regexs identifying paths that will not be stripped (matching is done with \texttt{match}, not \texttt{fullmatch}!) \\
		
		packaging\_hints & List(Tuple(str, str|List(str))) & [(bpkg, regex | [regex])] \\
		maint\_gen\_systemd & str & options to the maintainer script generator for systemd units \\
		
		dev\_dependencies & str & Generate dependencies between \texttt{-dev}-packages. Can only by \texttt{cdeps\_headers} for now. \\
	\end{tabularx}

	\subsection{\texttt{additional\_rdeps}}
	\label{sec:additional_rdeps}
	
	A version of \texttt{current} (will be translated to \texttt{2184801924} by \python{tslb.VersionNumber}) will be substituted with the binary version of the dependency, which is installed in the rootfs image during compilation. In a similar way \texttt{built} will be substituted by the currently built version number (for internal dependencies).
	
	\subsection{\texttt{packaging\_hints}}
	\label{sec:packaging_hints}
	
	One can give the build system hints about which files should be placed in which packages, or e.g. in separate packages. E.g. one can specify that a program 'a\_server' is placed in a separate package (maybe call it 'a\_server', too) instead of the generic binary package to separate server- and client implementations. The build system can then move files it associates with the program to the same package or its debug information to a designated \texttt{-dbgsym} package. The \texttt{packaging\_hints} are meant to make packaging such source packages easier compared to having to move every single file with \texttt{additional\_file\_placement}. They are evaluated first, which means the output produced can still be overriden with \texttt{additional\_file\_placement}.
	
	Regarding the parameter value format it is only required to associate file patterns ('fullmatch' python regular expressions) with binary package names, which can be done by a function paths $\mapsto$ binary packages or binary packages $\mapsto$ subsets of paths. I use the latter because the management shell's editor is already capable of editing such properties.
	
	The list is processed in order, files that have been assigned to a package already won't be moved to another package referenced later in the list, even if the regular expressions of that package match the file, too.
	
	\file{.dbg}-files are not moved when matched by patterns specified as hints, because debug packages will be derived from the elf-file placement in packages.
	
	Currently, the following actions are performed based on hints (apart from that they just move files, but before moving any other file which can lead to lower quality packaging if specific files are specified in \texttt{packaging\_hints}. E.g. perl's \file{.packlist}-files are used to identify perl packages, which requires them to be still in place at that point in the splitting-stage):
	
	\begin{itemize}
		\item For packages created based on hints that contain ELF files with GNU debug links, \texttt{-dbgsym}-packages are created automatically and files assigned.
		
		\item If any non-development file of a shared library is moved, all non-development files are moved to the new package (e.g. moving libreadline.so.8 will also move libreadline.so.8.1 or (slightly more interesting) moving \file{libnss\_files.so.2} will also move \file{libnss\_file-2.33.so}). And \file{.dbg}-files will be placed in corresponding \texttt{-dbgsym}-packages.
	\end{itemize}

	\subsection{\texttt{maint\_gen\_systemd}}
	\label{sec:maint_gen_systemd}
	
	By specifying '\texttt{disable}' one can prevent maintainer scripts for enabling etc. systemd services to be generated for \textit{all} binary packages created out of the source package.
	
	This attribute can also be a string-encoded JSON object as described in figure \ref{fig:maint_gen_systemd_json}, where all keys are optional.
	
	\begin{figure}[H]
		\centering
		\begin{minipage}{0.8\textwidth}
			\begin{lstlisting}
			{
				"units": {
					"<unit>": {
						"enable_on_install": false // defaults to true
					}
				}
			}
			\end{lstlisting}
		\end{minipage}
		
		\caption{\texttt{file\_splitter} stdin json-schema}
		\label{fig:maint_gen_systemd_json}
	\end{figure}
	
	
	\subsection{\texttt{file\_splitter}}
	\label{sec:file_splitter}
	
	The \texttt{file\_splitter} attribute allows to split the source package's installed files into arbitrary binary packages. It specifies an executable (usually some sort of script e.g. written in python). The executable receives a map of file associations that have been found so far encoded using json at stdin and is expected to write an updated map, along with other data do set on binary packages, to stdout. It is run after the regular file splitting facilities such that it can override the map and is provided with the file splitting logic's findings. The attribute can also contain a commandline to execute, because it uses the same running logic as build commands.
	
	If it is a script, it must carry a 'shebang' interpreter request. It is executed in the rootfs of the package currently being built using chroot s.t. it cannot destroy the build hosts filesystem accidentally, however that is the only protective measure and hence the executables are considered trusted. Moreover all software installed in the rootfs (e.g. using the \texttt{tools} attribute) is available to the executable.
	
	The format of the json-encoded data sent to stdin is sketched in figure \ref{fig:file_splitter_stdin_schema}. It is sent through a pipe, which is closed after all data has been sent hence the executable can determine that it received all data if it reads an EOF condition. The output to write to stdout follows the schema described by figure \ref{fig:file_splitter_stdout_schema}. Output to stderr is redirected to the build system's output and hence also included in the build state log. If the executable exits with a non-zero status, it is considered failed and the build system aborts the build of the package. The executable is run in the directory where the package has been installed (''destdir''), and additionally this directory is provided in the input data as well.
	
	Note that the file splitter can place a file into multiple (then conflicting) binary packages or assign no files to a binary package (and thus can create empty packages). Moreover it is not required to read stdin.
	
	\begin{figure}[htp]
		\centering
		
		\begin{minipage}{0.8\textwidth}
			\begin{lstlisting}
			{
				"source_package_name": <:str>,
				"install_root": <'destdir':str>,
				"architecture": <:str, e.g. 'amd64'>,
				"rootfs_installed_packages": [
					[<name:str>, <architecture:str, e.g. 'amd64'>, <version:str>]
				],
				"package_file_map": {
					"<binary package:str>": [<package-relative paths:str>],
				}
			}
			\end{lstlisting}
		\end{minipage}
		
		\caption{\texttt{file\_splitter} stdin json-schema}
		\label{fig:file_splitter_stdin_schema}
	\end{figure}

	\begin{figure}[htp]
		\centering
		
		\begin{minipage}{0.8\textwidth}
			\begin{lstlisting}
			{
				"package_file_map": {
					"<binary package:str>": [<package-relative paths:str>],
				},
				"set_attributes": {						// optional
					"<binary package:str>": [
						[<key:str>, <value:str>],
						[<key:str>, 'p', <base64 encoded pickled data>]
					]
				}
			}
			\end{lstlisting}
		\end{minipage}
		
		\caption{\texttt{file\_splitter} stdout json-schema}
		\label{fig:file_splitter_stdout_schema}
	\end{figure}

	The file splitter is completely free in which binary packages it generates. However a \texttt{'-all'} package should be generated (in fact the build system enforces this) s.t. the source package can later be installed as compiletime dependency. Moreover all packages for which attributes are added must exist in the \texttt{package\_file\_map}, even if no files are assigned to them.
	
\end{document}